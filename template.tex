% Options for packages loaded elsewhere
\PassOptionsToPackage{unicode}{hyperref}
\PassOptionsToPackage{hyphens}{url}
\PassOptionsToPackage{dvipsnames,svgnames,x11names}{xcolor}
%
\documentclass[
  11pt,
]{article}

\usepackage{amsmath,amssymb}
\usepackage{iftex}
\ifPDFTeX
  \usepackage[T1]{fontenc}
  \usepackage[utf8]{inputenc}
  \usepackage{textcomp} % provide euro and other symbols
\else % if luatex or xetex
  \usepackage{unicode-math}
  \defaultfontfeatures{Scale=MatchLowercase}
  \defaultfontfeatures[\rmfamily]{Ligatures=TeX,Scale=1}
\fi
\usepackage{lmodern}
\ifPDFTeX\else  
    % xetex/luatex font selection
\fi
% Use upquote if available, for straight quotes in verbatim environments
\IfFileExists{upquote.sty}{\usepackage{upquote}}{}
\IfFileExists{microtype.sty}{% use microtype if available
  \usepackage[]{microtype}
  \UseMicrotypeSet[protrusion]{basicmath} % disable protrusion for tt fonts
}{}
\makeatletter
\@ifundefined{KOMAClassName}{% if non-KOMA class
  \IfFileExists{parskip.sty}{%
    \usepackage{parskip}
  }{% else
    \setlength{\parindent}{0pt}
    \setlength{\parskip}{6pt plus 2pt minus 1pt}}
}{% if KOMA class
  \KOMAoptions{parskip=half}}
\makeatother
\usepackage{xcolor}
\usepackage[margin=1in]{geometry}
\setlength{\emergencystretch}{3em} % prevent overfull lines
\setcounter{secnumdepth}{5}
% Make \paragraph and \subparagraph free-standing
\ifx\paragraph\undefined\else
  \let\oldparagraph\paragraph
  \renewcommand{\paragraph}[1]{\oldparagraph{#1}\mbox{}}
\fi
\ifx\subparagraph\undefined\else
  \let\oldsubparagraph\subparagraph
  \renewcommand{\subparagraph}[1]{\oldsubparagraph{#1}\mbox{}}
\fi

\usepackage{color}
\usepackage{fancyvrb}
\newcommand{\VerbBar}{|}
\newcommand{\VERB}{\Verb[commandchars=\\\{\}]}
\DefineVerbatimEnvironment{Highlighting}{Verbatim}{commandchars=\\\{\}}
% Add ',fontsize=\small' for more characters per line
\usepackage{framed}
\definecolor{shadecolor}{RGB}{241,243,245}
\newenvironment{Shaded}{\begin{snugshade}}{\end{snugshade}}
\newcommand{\AlertTok}[1]{\textcolor[rgb]{0.68,0.00,0.00}{#1}}
\newcommand{\AnnotationTok}[1]{\textcolor[rgb]{0.37,0.37,0.37}{#1}}
\newcommand{\AttributeTok}[1]{\textcolor[rgb]{0.40,0.45,0.13}{#1}}
\newcommand{\BaseNTok}[1]{\textcolor[rgb]{0.68,0.00,0.00}{#1}}
\newcommand{\BuiltInTok}[1]{\textcolor[rgb]{0.00,0.23,0.31}{#1}}
\newcommand{\CharTok}[1]{\textcolor[rgb]{0.13,0.47,0.30}{#1}}
\newcommand{\CommentTok}[1]{\textcolor[rgb]{0.37,0.37,0.37}{#1}}
\newcommand{\CommentVarTok}[1]{\textcolor[rgb]{0.37,0.37,0.37}{\textit{#1}}}
\newcommand{\ConstantTok}[1]{\textcolor[rgb]{0.56,0.35,0.01}{#1}}
\newcommand{\ControlFlowTok}[1]{\textcolor[rgb]{0.00,0.23,0.31}{#1}}
\newcommand{\DataTypeTok}[1]{\textcolor[rgb]{0.68,0.00,0.00}{#1}}
\newcommand{\DecValTok}[1]{\textcolor[rgb]{0.68,0.00,0.00}{#1}}
\newcommand{\DocumentationTok}[1]{\textcolor[rgb]{0.37,0.37,0.37}{\textit{#1}}}
\newcommand{\ErrorTok}[1]{\textcolor[rgb]{0.68,0.00,0.00}{#1}}
\newcommand{\ExtensionTok}[1]{\textcolor[rgb]{0.00,0.23,0.31}{#1}}
\newcommand{\FloatTok}[1]{\textcolor[rgb]{0.68,0.00,0.00}{#1}}
\newcommand{\FunctionTok}[1]{\textcolor[rgb]{0.28,0.35,0.67}{#1}}
\newcommand{\ImportTok}[1]{\textcolor[rgb]{0.00,0.46,0.62}{#1}}
\newcommand{\InformationTok}[1]{\textcolor[rgb]{0.37,0.37,0.37}{#1}}
\newcommand{\KeywordTok}[1]{\textcolor[rgb]{0.00,0.23,0.31}{#1}}
\newcommand{\NormalTok}[1]{\textcolor[rgb]{0.00,0.23,0.31}{#1}}
\newcommand{\OperatorTok}[1]{\textcolor[rgb]{0.37,0.37,0.37}{#1}}
\newcommand{\OtherTok}[1]{\textcolor[rgb]{0.00,0.23,0.31}{#1}}
\newcommand{\PreprocessorTok}[1]{\textcolor[rgb]{0.68,0.00,0.00}{#1}}
\newcommand{\RegionMarkerTok}[1]{\textcolor[rgb]{0.00,0.23,0.31}{#1}}
\newcommand{\SpecialCharTok}[1]{\textcolor[rgb]{0.37,0.37,0.37}{#1}}
\newcommand{\SpecialStringTok}[1]{\textcolor[rgb]{0.13,0.47,0.30}{#1}}
\newcommand{\StringTok}[1]{\textcolor[rgb]{0.13,0.47,0.30}{#1}}
\newcommand{\VariableTok}[1]{\textcolor[rgb]{0.07,0.07,0.07}{#1}}
\newcommand{\VerbatimStringTok}[1]{\textcolor[rgb]{0.13,0.47,0.30}{#1}}
\newcommand{\WarningTok}[1]{\textcolor[rgb]{0.37,0.37,0.37}{\textit{#1}}}

\providecommand{\tightlist}{%
  \setlength{\itemsep}{0pt}\setlength{\parskip}{0pt}}\usepackage{longtable,booktabs,array}
\usepackage{calc} % for calculating minipage widths
% Correct order of tables after \paragraph or \subparagraph
\usepackage{etoolbox}
\makeatletter
\patchcmd\longtable{\par}{\if@noskipsec\mbox{}\fi\par}{}{}
\makeatother
% Allow footnotes in longtable head/foot
\IfFileExists{footnotehyper.sty}{\usepackage{footnotehyper}}{\usepackage{footnote}}
\makesavenoteenv{longtable}
\usepackage{graphicx}
\makeatletter
\def\maxwidth{\ifdim\Gin@nat@width>\linewidth\linewidth\else\Gin@nat@width\fi}
\def\maxheight{\ifdim\Gin@nat@height>\textheight\textheight\else\Gin@nat@height\fi}
\makeatother
% Scale images if necessary, so that they will not overflow the page
% margins by default, and it is still possible to overwrite the defaults
% using explicit options in \includegraphics[width, height, ...]{}
\setkeys{Gin}{width=\maxwidth,height=\maxheight,keepaspectratio}
% Set default figure placement to htbp
\makeatletter
\def\fps@figure{htbp}
\makeatother
% definitions for citeproc citations
\NewDocumentCommand\citeproctext{}{}
\NewDocumentCommand\citeproc{mm}{%
  \begingroup\def\citeproctext{#2}\cite{#1}\endgroup}
\makeatletter
 % allow citations to break across lines
 \let\@cite@ofmt\@firstofone
 % avoid brackets around text for \cite:
 \def\@biblabel#1{}
 \def\@cite#1#2{{#1\if@tempswa , #2\fi}}
\makeatother
\newlength{\cslhangindent}
\setlength{\cslhangindent}{1.5em}
\newlength{\csllabelwidth}
\setlength{\csllabelwidth}{3em}
\newenvironment{CSLReferences}[2] % #1 hanging-indent, #2 entry-spacing
 {\begin{list}{}{%
  \setlength{\itemindent}{0pt}
  \setlength{\leftmargin}{0pt}
  \setlength{\parsep}{0pt}
  % turn on hanging indent if param 1 is 1
  \ifodd #1
   \setlength{\leftmargin}{\cslhangindent}
   \setlength{\itemindent}{-1\cslhangindent}
  \fi
  % set entry spacing
  \setlength{\itemsep}{#2\baselineskip}}}
 {\end{list}}
\usepackage{calc}
\newcommand{\CSLBlock}[1]{\hfill\break\parbox[t]{\linewidth}{\strut\ignorespaces#1\strut}}
\newcommand{\CSLLeftMargin}[1]{\parbox[t]{\csllabelwidth}{\strut#1\strut}}
\newcommand{\CSLRightInline}[1]{\parbox[t]{\linewidth - \csllabelwidth}{\strut#1\strut}}
\newcommand{\CSLIndent}[1]{\hspace{\cslhangindent}#1}

\makeatletter
\@ifpackageloaded{caption}{}{\usepackage{caption}}
\AtBeginDocument{%
\ifdefined\contentsname
  \renewcommand*\contentsname{Table of contents}
\else
  \newcommand\contentsname{Table of contents}
\fi
\ifdefined\listfigurename
  \renewcommand*\listfigurename{List of Figures}
\else
  \newcommand\listfigurename{List of Figures}
\fi
\ifdefined\listtablename
  \renewcommand*\listtablename{List of Tables}
\else
  \newcommand\listtablename{List of Tables}
\fi
\ifdefined\figurename
  \renewcommand*\figurename{Figure}
\else
  \newcommand\figurename{Figure}
\fi
\ifdefined\tablename
  \renewcommand*\tablename{Table}
\else
  \newcommand\tablename{Table}
\fi
}
\@ifpackageloaded{float}{}{\usepackage{float}}
\floatstyle{ruled}
\@ifundefined{c@chapter}{\newfloat{codelisting}{h}{lop}}{\newfloat{codelisting}{h}{lop}[chapter]}
\floatname{codelisting}{Listing}
\newcommand*\listoflistings{\listof{codelisting}{List of Listings}}
\makeatother
\makeatletter
\makeatother
\makeatletter
\@ifpackageloaded{caption}{}{\usepackage{caption}}
\@ifpackageloaded{subcaption}{}{\usepackage{subcaption}}
\makeatother
\ifLuaTeX
  \usepackage{selnolig}  % disable illegal ligatures
\fi
\usepackage{bookmark}

\IfFileExists{xurl.sty}{\usepackage{xurl}}{} % add URL line breaks if available
\urlstyle{same} % disable monospaced font for URLs
\hypersetup{
  pdftitle={Final Project Report},
  pdfauthor={Wanqi Yuan (wy21) \& Jonathan Gan (wg18)},
  colorlinks=true,
  linkcolor={blue},
  filecolor={Maroon},
  citecolor={Blue},
  urlcolor={Blue},
  pdfcreator={LaTeX via pandoc}}

\title{Final Project Report}
\author{Wanqi Yuan (wy21) \& Jonathan Gan (wg18)}
\date{Tue., Apr.~30}

\begin{document}
\maketitle

\section{Introduction}\label{introduction}

\subsection{Problem Statement}\label{problem-statement}

The parking garage problem is currently structured to assume that demand
is deterministic and constant. This is not representative of real world
conditions as demand is highly dynamic and will change in response to
various city conditions. For our final project, we want to better
capture the variance in demand for parking, particularly in Houston. We
will explore through research the relationship between price of parking
and the demand of parking to implement into our get\_action function. We
will mathematically represent this relationship through a equation
derived from our research. We will be considering the case where we are
taking yearly time steps to represent a garage that rents out spots on a
yearly basis. At the end of each year, we will evaluate whether demand
of that year given the elasticity of demand, exceeds capacity enough to
justify building an additional level. We will analyze this case for
multiple demand vs.~price curve.

Demand is a quantifiable variable for many cases outside of parking that
determine whether a given action will be taken. By more accurately
modeling this relationship between how the price of \emph{something}
might change the demand of it, and therefore the construction that might
take place as a result, we can apply this to other climate scenarios
that would require sequential decision making. For example, the decision
to expand a solar farm on a an annual basis dependent on the demand of
the energy for the clients it serve.

\subsection{Selected Feature}\label{selected-feature}

Describe the feature you have selected to add to the existing
decision-support tool. Discuss how this feature relates to the problem
statement and its potential to improve climate risk assessment.

\section{Literature Review}\label{literature-review}

Provide a brief overview of the theoretical background related to your
chosen feature. Cite at least two relevant journal articles to support
your approach (see
\href{https://quarto.org/docs/authoring/footnotes-and-citations.html}{Quarto
docs} for help with citations). Explain how these articles contribute to
the justification of your selected feature.

\section{Methodology}\label{methodology}

\subsection{Implementation}\label{implementation}

You should make your modifications in either the \texttt{HouseElevation}
or \texttt{ParkingGarage} module. Detail the steps taken to implement
the selected feature and integrate it into the decision-support tool.
Include code snippets and explanations where necessary to clarify the
implementation process.

\subsection{Validation}\label{validation}

As we have seen in labs, mistakes are inevitable and can lead to
misleading results. To minimize the risk of errors making their way into
final results, it is essential to validate the implemented feature.
Describe the validation techniques used to ensure the accuracy and
reliability of your implemented feature. Discuss any challenges faced
during the validation process and how they were addressed.

\section{Results}\label{results}

Present the results obtained from the enhanced decision-support tool.
Use tables, figures, and visualizations to clearly communicate the
outcomes. Provide sufficient detail to demonstrate how the implemented
feature addresses the problem statement. Use the
\texttt{\#\textbar{}\ output:\ false} and/or
\texttt{\#\textbar{}\ echo:\ false} tags to hide code output and code
cells in the final report except where showing the output (e.g.g, a
plot) or the code (e.g., how you are sampling SOWs) adds value to the
discussion. You may have multiple subsections of results, which you can
create using \texttt{\#\#}.

\section{Conclusions}\label{conclusions}

\subsection{Discussion}\label{discussion}

Analyze the implications of your results for climate risk management.
Consider the context of the class themes and discuss how your findings
contribute to the understanding of climate risk assessment. Identify any
limitations of your approach and suggest potential improvements for
future work.

\subsection{Conclusions}\label{conclusions-1}

Summarize the key findings of your project and reiterate the
significance of your implemented feature in addressing the problem
statement. Discuss the broader implications of your work for climate
risk management and the potential for further research in this area.

\section{References}\label{references}

\phantomsection\label{refs}
\begin{CSLReferences}{0}{1}
\end{CSLReferences}

\begin{Shaded}
\begin{Highlighting}[numbers=left,,]
\ImportTok{using} \BuiltInTok{Plots}
\ImportTok{using} \BuiltInTok{Distributions}
\NormalTok{Plots.}\FunctionTok{default}\NormalTok{(; margin}\OperatorTok{=}\FloatTok{5}\NormalTok{Plots.mm)}
\end{Highlighting}
\end{Shaded}

\begin{Shaded}
\begin{Highlighting}[numbers=left,,]
\KeywordTok{function} \FunctionTok{get\_profit}\NormalTok{(capacity, m, b, cost)}

    \CommentTok{\# Cost is cost per spot }
\NormalTok{    price }\OperatorTok{=} \FunctionTok{div}\NormalTok{(capacity }\OperatorTok{{-}}\NormalTok{ b, m) }\CommentTok{\# Floor Div for dollars}
    \ControlFlowTok{return}\NormalTok{ (price }\OperatorTok{*}\NormalTok{ capacity)}\FunctionTok{{-}}\NormalTok{(cost }\OperatorTok{*}\NormalTok{ capacity)}
\KeywordTok{end}
\end{Highlighting}
\end{Shaded}

\begin{verbatim}
get_profit (generic function with 1 method)
\end{verbatim}

\begin{Shaded}
\begin{Highlighting}[numbers=left,,]
\KeywordTok{function} \FunctionTok{optimize}\NormalTok{(maxSpots, increment, start\_spots, m, cost\_per\_spot)}
\NormalTok{    b }\OperatorTok{=}\NormalTok{ maxSpots}
\NormalTok{    levels\_to\_add }\OperatorTok{=} \FunctionTok{div}\NormalTok{(maxSpots }\OperatorTok{{-}}\NormalTok{ start\_spots, increment)}
\NormalTok{    optLevels }\OperatorTok{=} \FloatTok{0} 
\NormalTok{    optProfit }\OperatorTok{=} \FunctionTok{get\_profit}\NormalTok{(start\_spots, m, b, cost\_per\_spot)}
\NormalTok{    curr\_levels }\OperatorTok{=} \FloatTok{0}
    \ControlFlowTok{while}\NormalTok{ curr\_levels }\OperatorTok{\textless{}}\NormalTok{ levels\_to\_add}
\NormalTok{        curr\_capacity }\OperatorTok{=}\NormalTok{ start\_spots }\OperatorTok{+}\NormalTok{ curr\_levels }\OperatorTok{*}\NormalTok{ increment}
\NormalTok{        profit }\OperatorTok{=} \FunctionTok{get\_profit}\NormalTok{(curr\_capacity, m, b, cost\_per\_spot)}
        \ControlFlowTok{if}\NormalTok{ profit }\OperatorTok{\textgreater{}}\NormalTok{ optProfit}
\NormalTok{            optProfit }\OperatorTok{=}\NormalTok{ profit}
\NormalTok{            optLevels }\OperatorTok{=}\NormalTok{ curr\_levels}
        \ControlFlowTok{end}
\NormalTok{        curr\_levels }\OperatorTok{+=} \FloatTok{1}
    \ControlFlowTok{end}
    \ControlFlowTok{return}\NormalTok{ optLevels}
\KeywordTok{end}
\end{Highlighting}
\end{Shaded}

\begin{verbatim}
optimize (generic function with 1 method)
\end{verbatim}

\begin{Shaded}
\begin{Highlighting}[numbers=left,,]
\KeywordTok{function} \FunctionTok{simulation}\NormalTok{(timesteps, start\_spots, increment, demand\_growth\_rate, init\_b, m, inflation\_rate, cost\_per\_spot)}
\NormalTok{    y\_levels }\OperatorTok{=}\NormalTok{ [] }\CommentTok{\# LEVELS AT TIMETSTEP T}
\NormalTok{    y\_curr\_profit }\OperatorTok{=}\NormalTok{ [] }\CommentTok{\# PROFIT AT TIMESTEP T}
\NormalTok{    y\_net\_profit }\OperatorTok{=}\NormalTok{ [] }\CommentTok{\# NET PROFITS}
\NormalTok{    net\_profit }\OperatorTok{=} \FloatTok{0}
\NormalTok{    curr\_levels }\OperatorTok{=} \FloatTok{0}
\NormalTok{    timestep }\OperatorTok{=} \FloatTok{0}
\NormalTok{    curr\_spots }\OperatorTok{=}\NormalTok{ start\_spots}
\NormalTok{    b }\OperatorTok{=}\NormalTok{ init\_b}

    \ControlFlowTok{while}\NormalTok{ timestep }\OperatorTok{\textless{}}\NormalTok{ timesteps}
\NormalTok{        levels\_to\_add }\OperatorTok{=} \FunctionTok{optimize}\NormalTok{(b, increment, curr\_spots, m, cost\_per\_spot)}
        \ControlFlowTok{if}\NormalTok{ levels\_to\_add }\OperatorTok{==} \FloatTok{0}
            \CommentTok{\# println("Whoops")}
        \ControlFlowTok{end}
\NormalTok{        curr\_spots }\OperatorTok{+=}\NormalTok{ increment }\OperatorTok{*}\NormalTok{ levels\_to\_add}
\NormalTok{        curr\_levels }\OperatorTok{+=}\NormalTok{ levels\_to\_add}

\NormalTok{        curr\_profit }\OperatorTok{=} \FunctionTok{get\_profit}\NormalTok{(curr\_spots, m, b, cost\_per\_spot) }\OperatorTok{/} \FloatTok{1000}
\NormalTok{        net\_profit }\OperatorTok{+=}\NormalTok{ curr\_profit}
        \FunctionTok{push!}\NormalTok{(y\_curr\_profit, curr\_profit)}
        \FunctionTok{push!}\NormalTok{(y\_net\_profit, net\_profit)}
        \FunctionTok{push!}\NormalTok{(y\_levels, curr\_levels)}

\NormalTok{        b }\OperatorTok{*=}\NormalTok{ (}\FloatTok{1} \OperatorTok{+}\NormalTok{ demand\_growth\_rate)}
\NormalTok{        m }\OperatorTok{/=}\NormalTok{ (}\FloatTok{1} \OperatorTok{+}\NormalTok{ inflation\_rate)}

\NormalTok{        timestep }\OperatorTok{+=} \FloatTok{1} 
    \ControlFlowTok{end}

    \ControlFlowTok{return}\NormalTok{ y\_levels}
\KeywordTok{end}
\end{Highlighting}
\end{Shaded}

\begin{verbatim}
simulation (generic function with 1 method)
\end{verbatim}

INCREMENT = 100 Inflation Rate = 0.04 Construction Cost = 30000

\begin{Shaded}
\begin{Highlighting}[numbers=left,,]
\NormalTok{timesteps }\OperatorTok{=} \FloatTok{25} 
\NormalTok{start\_spots }\OperatorTok{=} \FloatTok{0}
\NormalTok{increment }\OperatorTok{=} \FloatTok{100}
\NormalTok{demand\_growth\_rate }\OperatorTok{=} \FunctionTok{rand}\NormalTok{(}\FunctionTok{Normal}\NormalTok{(}\FloatTok{0.0045}\NormalTok{, }\FloatTok{0.001}\NormalTok{))}
\NormalTok{init\_b }\OperatorTok{=} \FloatTok{1500}
\NormalTok{m }\OperatorTok{=} \OperatorTok{{-}}\FloatTok{0.046}
\NormalTok{inflation\_rate }\OperatorTok{=} \FloatTok{0.04}
\NormalTok{cost\_per\_spot }\OperatorTok{=} \FloatTok{30000}
\NormalTok{y\_levels }\OperatorTok{=} \FunctionTok{simulation}\NormalTok{(timesteps, start\_spots, increment, demand\_growth\_rate, init\_b, m, inflation\_rate, cost\_per\_spot)}
\NormalTok{display\_growth\_rate }\OperatorTok{=} \FunctionTok{round}\NormalTok{(demand\_growth\_rate, sigdigits}\OperatorTok{=}\FloatTok{3}\NormalTok{)}
\NormalTok{plot\_title }\OperatorTok{=} \StringTok{"Parking Garage Simulation"}
\NormalTok{font\_size }\OperatorTok{=} \FloatTok{16}
\FunctionTok{println}\NormalTok{(y\_levels)}
\FunctionTok{plot}\NormalTok{(y\_levels, label}\OperatorTok{=}\StringTok{"Optimal Levels at Time t"}\NormalTok{, xlabel}\OperatorTok{=}\StringTok{"Year"}\NormalTok{, ylabel}\OperatorTok{=}\StringTok{"Number of Levels"}\NormalTok{,}
\NormalTok{title}\OperatorTok{=}\NormalTok{plot\_title, titlefontsize}\OperatorTok{=}\NormalTok{font\_size, lw}\OperatorTok{=}\FloatTok{1}\NormalTok{, gridlinewidth}\OperatorTok{=}\FloatTok{1}\NormalTok{, xticks}\OperatorTok{=}\FloatTok{0}\OperatorTok{:}\FloatTok{1}\OperatorTok{:}\FloatTok{25}\NormalTok{, yticks}\OperatorTok{=}\FloatTok{0}\OperatorTok{:}\FloatTok{1}\OperatorTok{:}\FloatTok{100}\NormalTok{)}
\CommentTok{\# plot(y\_curr\_profit, label="Current Annual Profit at time t")}
\CommentTok{\# plot(y\_net\_profit, label="Net Profit at time t")}
\CommentTok{\# legend()}
\CommentTok{\#display()}
\end{Highlighting}
\end{Shaded}

\begin{verbatim}
Any[1, 1, 1, 1, 2, 2, 2, 2, 3, 3, 3, 3, 3, 4, 4, 4, 4, 4, 4, 5, 5, 5, 5, 5, 5]
\end{verbatim}

\includegraphics{template_files/mediabag/template_files/figure-pdf/cell-6-output-2.pdf}

INCREMENT = 150 Inflation Rate = 0.04 Construction Cost = 30000

\begin{verbatim}
Any[0, 1, 1, 1, 1, 1, 2, 2, 2, 2, 2, 2, 2, 3, 3, 3, 3, 3, 3, 3, 3, 3, 4, 4, 4]
\end{verbatim}

\includegraphics{template_files/mediabag/template_files/figure-pdf/cell-7-output-2.pdf}

INCREMENT = 200 Inflation Rate = 0.04 Construction Cost = 30000

\begin{verbatim}
Any[0, 0, 1, 1, 1, 1, 1, 1, 1, 2, 2, 2, 2, 2, 2, 2, 2, 2, 2, 3, 3, 3, 3, 3, 3]
\end{verbatim}

\includegraphics{template_files/mediabag/template_files/figure-pdf/cell-8-output-2.pdf}

Increment = 150 INFLATION RATE = 0.03 Construction Cost = 30000

\begin{Shaded}
\begin{Highlighting}[numbers=left,,]
\NormalTok{timesteps }\OperatorTok{=} \FloatTok{25} 
\NormalTok{start\_spots }\OperatorTok{=} \FloatTok{0}
\NormalTok{increment }\OperatorTok{=} \FloatTok{150}
\NormalTok{demand\_growth\_rate }\OperatorTok{=} \FunctionTok{rand}\NormalTok{(}\FunctionTok{Normal}\NormalTok{(}\FloatTok{0.0045}\NormalTok{, }\FloatTok{0.001}\NormalTok{))}
\NormalTok{init\_b }\OperatorTok{=} \FloatTok{1500}
\NormalTok{m }\OperatorTok{=} \OperatorTok{{-}}\FloatTok{0.046}
\NormalTok{inflation\_rate }\OperatorTok{=} \FloatTok{0.03}
\NormalTok{cost\_per\_spot }\OperatorTok{=} \FloatTok{30000}
\NormalTok{y\_levels }\OperatorTok{=} \FunctionTok{simulation}\NormalTok{(timesteps, start\_spots, increment, demand\_growth\_rate, init\_b, m, inflation\_rate, cost\_per\_spot)}
\NormalTok{display\_growth\_rate }\OperatorTok{=} \FunctionTok{round}\NormalTok{(demand\_growth\_rate, sigdigits}\OperatorTok{=}\FloatTok{3}\NormalTok{)}
\NormalTok{plot\_title }\OperatorTok{=} \StringTok{"Parking Garage Simulation"}
\NormalTok{font\_size }\OperatorTok{=} \FloatTok{16}
\FunctionTok{println}\NormalTok{(y\_levels)}
\FunctionTok{plot}\NormalTok{(y\_levels, label}\OperatorTok{=}\StringTok{"Optimal Levels at Time t"}\NormalTok{, xlabel}\OperatorTok{=}\StringTok{"Year"}\NormalTok{, ylabel}\OperatorTok{=}\StringTok{"Number of Levels"}\NormalTok{,}
\NormalTok{title}\OperatorTok{=}\NormalTok{plot\_title, titlefontsize}\OperatorTok{=}\NormalTok{font\_size, lw}\OperatorTok{=}\FloatTok{1}\NormalTok{, gridlinewidth}\OperatorTok{=}\FloatTok{1}\NormalTok{, xticks}\OperatorTok{=}\FloatTok{0}\OperatorTok{:}\FloatTok{1}\OperatorTok{:}\FloatTok{25}\NormalTok{, yticks}\OperatorTok{=}\FloatTok{0}\OperatorTok{:}\FloatTok{1}\OperatorTok{:}\FloatTok{100}\NormalTok{)}
\CommentTok{\# plot(y\_curr\_profit, label="Current Annual Profit at time t")}
\CommentTok{\# plot(y\_net\_profit, label="Net Profit at time t")}
\CommentTok{\# legend()}
\CommentTok{\#display()}
\end{Highlighting}
\end{Shaded}

\begin{verbatim}
Any[0, 1, 1, 1, 1, 1, 1, 1, 2, 2, 2, 2, 2, 2, 2, 2, 2, 3, 3, 3, 3, 3, 3, 3, 3]
\end{verbatim}

\includegraphics{template_files/mediabag/template_files/figure-pdf/cell-9-output-2.pdf}

Increment = 150 INFLATION RATE = 0.04 Construction Cost = 30000

\begin{verbatim}
Any[0, 1, 1, 1, 1, 1, 2, 2, 2, 2, 2, 2, 2, 3, 3, 3, 3, 3, 3, 3, 3, 4, 4, 4, 4]
\end{verbatim}

\includegraphics{template_files/mediabag/template_files/figure-pdf/cell-10-output-2.pdf}

Increment = 150 INFLATION RATE = 0.05 Construction Cost = 30000

\begin{verbatim}
Any[0, 1, 1, 1, 1, 2, 2, 2, 2, 2, 2, 3, 3, 3, 3, 3, 3, 3, 4, 4, 4, 4, 4, 4, 4]
\end{verbatim}

\includegraphics{template_files/mediabag/template_files/figure-pdf/cell-11-output-2.pdf}

Increment = 150 Inflation Rate = 0.04 CONSTRUCTION COST = 20000

\begin{Shaded}
\begin{Highlighting}[numbers=left,,]
\NormalTok{timesteps }\OperatorTok{=} \FloatTok{25} 
\NormalTok{start\_spots }\OperatorTok{=} \FloatTok{0}
\NormalTok{increment }\OperatorTok{=} \FloatTok{150}
\NormalTok{demand\_growth\_rate }\OperatorTok{=} \FunctionTok{rand}\NormalTok{(}\FunctionTok{Normal}\NormalTok{(}\FloatTok{0.0045}\NormalTok{, }\FloatTok{0.001}\NormalTok{))}
\NormalTok{init\_b }\OperatorTok{=} \FloatTok{1500}
\NormalTok{m }\OperatorTok{=} \OperatorTok{{-}}\FloatTok{0.046}
\NormalTok{inflation\_rate }\OperatorTok{=} \FloatTok{0.04}
\NormalTok{cost\_per\_spot }\OperatorTok{=} \FloatTok{20000}
\NormalTok{y\_levels }\OperatorTok{=} \FunctionTok{simulation}\NormalTok{(timesteps, start\_spots, increment, demand\_growth\_rate, init\_b, m, inflation\_rate, cost\_per\_spot)}
\NormalTok{display\_growth\_rate }\OperatorTok{=} \FunctionTok{round}\NormalTok{(demand\_growth\_rate, sigdigits}\OperatorTok{=}\FloatTok{3}\NormalTok{)}
\NormalTok{plot\_title }\OperatorTok{=} \StringTok{"Parking Garage Simulation"}
\NormalTok{font\_size }\OperatorTok{=} \FloatTok{16}
\FunctionTok{println}\NormalTok{(y\_levels)}
\FunctionTok{plot}\NormalTok{(y\_levels, label}\OperatorTok{=}\StringTok{"Optimal Levels at Time t"}\NormalTok{, xlabel}\OperatorTok{=}\StringTok{"Year"}\NormalTok{, ylabel}\OperatorTok{=}\StringTok{"Number of Levels"}\NormalTok{,}
\NormalTok{title}\OperatorTok{=}\NormalTok{plot\_title, titlefontsize}\OperatorTok{=}\NormalTok{font\_size, lw}\OperatorTok{=}\FloatTok{1}\NormalTok{, gridlinewidth}\OperatorTok{=}\FloatTok{1}\NormalTok{, xticks}\OperatorTok{=}\FloatTok{0}\OperatorTok{:}\FloatTok{1}\OperatorTok{:}\FloatTok{25}\NormalTok{, yticks}\OperatorTok{=}\FloatTok{0}\OperatorTok{:}\FloatTok{1}\OperatorTok{:}\FloatTok{100}\NormalTok{)}
\CommentTok{\# plot(y\_curr\_profit, label="Current Annual Profit at time t")}
\CommentTok{\# plot(y\_net\_profit, label="Net Profit at time t")}
\CommentTok{\# legend()}
\CommentTok{\#display()}
\end{Highlighting}
\end{Shaded}

\begin{verbatim}
Any[2, 2, 2, 2, 2, 3, 3, 3, 3, 3, 3, 3, 3, 4, 4, 4, 4, 4, 4, 4, 4, 4, 4, 4, 4]
\end{verbatim}

\includegraphics{template_files/mediabag/template_files/figure-pdf/cell-12-output-2.pdf}

Increment = 150 Inflation Rate = 0.04 CONSTRUCTION COST = 30000

\begin{verbatim}
Any[0, 1, 1, 1, 1, 1, 2, 2, 2, 2, 2, 2, 2, 3, 3, 3, 3, 3, 3, 3, 3, 4, 4, 4, 4]
\end{verbatim}

\includegraphics{template_files/mediabag/template_files/figure-pdf/cell-13-output-2.pdf}

Increment = 150 Inflation Rate = 0.04 CONSTRUCTION COST = 40000

\begin{verbatim}
Any[0, 0, 0, 0, 0, 0, 0, 0, 1, 1, 1, 1, 1, 2, 2, 2, 2, 2, 2, 3, 3, 3, 3, 3, 3]
\end{verbatim}

\includegraphics{template_files/mediabag/template_files/figure-pdf/cell-14-output-2.pdf}



\end{document}
