% Options for packages loaded elsewhere
\PassOptionsToPackage{unicode}{hyperref}
\PassOptionsToPackage{hyphens}{url}
\PassOptionsToPackage{dvipsnames,svgnames,x11names}{xcolor}
%
\documentclass[
  letterpaper,
  DIV=11,
  numbers=noendperiod]{scrartcl}

\usepackage{amsmath,amssymb}
\usepackage{iftex}
\ifPDFTeX
  \usepackage[T1]{fontenc}
  \usepackage[utf8]{inputenc}
  \usepackage{textcomp} % provide euro and other symbols
\else % if luatex or xetex
  \usepackage{unicode-math}
  \defaultfontfeatures{Scale=MatchLowercase}
  \defaultfontfeatures[\rmfamily]{Ligatures=TeX,Scale=1}
\fi
\usepackage{lmodern}
\ifPDFTeX\else  
    % xetex/luatex font selection
\fi
% Use upquote if available, for straight quotes in verbatim environments
\IfFileExists{upquote.sty}{\usepackage{upquote}}{}
\IfFileExists{microtype.sty}{% use microtype if available
  \usepackage[]{microtype}
  \UseMicrotypeSet[protrusion]{basicmath} % disable protrusion for tt fonts
}{}
\makeatletter
\@ifundefined{KOMAClassName}{% if non-KOMA class
  \IfFileExists{parskip.sty}{%
    \usepackage{parskip}
  }{% else
    \setlength{\parindent}{0pt}
    \setlength{\parskip}{6pt plus 2pt minus 1pt}}
}{% if KOMA class
  \KOMAoptions{parskip=half}}
\makeatother
\usepackage{xcolor}
\setlength{\emergencystretch}{3em} % prevent overfull lines
\setcounter{secnumdepth}{-\maxdimen} % remove section numbering
% Make \paragraph and \subparagraph free-standing
\ifx\paragraph\undefined\else
  \let\oldparagraph\paragraph
  \renewcommand{\paragraph}[1]{\oldparagraph{#1}\mbox{}}
\fi
\ifx\subparagraph\undefined\else
  \let\oldsubparagraph\subparagraph
  \renewcommand{\subparagraph}[1]{\oldsubparagraph{#1}\mbox{}}
\fi

\usepackage{color}
\usepackage{fancyvrb}
\newcommand{\VerbBar}{|}
\newcommand{\VERB}{\Verb[commandchars=\\\{\}]}
\DefineVerbatimEnvironment{Highlighting}{Verbatim}{commandchars=\\\{\}}
% Add ',fontsize=\small' for more characters per line
\usepackage{framed}
\definecolor{shadecolor}{RGB}{241,243,245}
\newenvironment{Shaded}{\begin{snugshade}}{\end{snugshade}}
\newcommand{\AlertTok}[1]{\textcolor[rgb]{0.68,0.00,0.00}{#1}}
\newcommand{\AnnotationTok}[1]{\textcolor[rgb]{0.37,0.37,0.37}{#1}}
\newcommand{\AttributeTok}[1]{\textcolor[rgb]{0.40,0.45,0.13}{#1}}
\newcommand{\BaseNTok}[1]{\textcolor[rgb]{0.68,0.00,0.00}{#1}}
\newcommand{\BuiltInTok}[1]{\textcolor[rgb]{0.00,0.23,0.31}{#1}}
\newcommand{\CharTok}[1]{\textcolor[rgb]{0.13,0.47,0.30}{#1}}
\newcommand{\CommentTok}[1]{\textcolor[rgb]{0.37,0.37,0.37}{#1}}
\newcommand{\CommentVarTok}[1]{\textcolor[rgb]{0.37,0.37,0.37}{\textit{#1}}}
\newcommand{\ConstantTok}[1]{\textcolor[rgb]{0.56,0.35,0.01}{#1}}
\newcommand{\ControlFlowTok}[1]{\textcolor[rgb]{0.00,0.23,0.31}{#1}}
\newcommand{\DataTypeTok}[1]{\textcolor[rgb]{0.68,0.00,0.00}{#1}}
\newcommand{\DecValTok}[1]{\textcolor[rgb]{0.68,0.00,0.00}{#1}}
\newcommand{\DocumentationTok}[1]{\textcolor[rgb]{0.37,0.37,0.37}{\textit{#1}}}
\newcommand{\ErrorTok}[1]{\textcolor[rgb]{0.68,0.00,0.00}{#1}}
\newcommand{\ExtensionTok}[1]{\textcolor[rgb]{0.00,0.23,0.31}{#1}}
\newcommand{\FloatTok}[1]{\textcolor[rgb]{0.68,0.00,0.00}{#1}}
\newcommand{\FunctionTok}[1]{\textcolor[rgb]{0.28,0.35,0.67}{#1}}
\newcommand{\ImportTok}[1]{\textcolor[rgb]{0.00,0.46,0.62}{#1}}
\newcommand{\InformationTok}[1]{\textcolor[rgb]{0.37,0.37,0.37}{#1}}
\newcommand{\KeywordTok}[1]{\textcolor[rgb]{0.00,0.23,0.31}{#1}}
\newcommand{\NormalTok}[1]{\textcolor[rgb]{0.00,0.23,0.31}{#1}}
\newcommand{\OperatorTok}[1]{\textcolor[rgb]{0.37,0.37,0.37}{#1}}
\newcommand{\OtherTok}[1]{\textcolor[rgb]{0.00,0.23,0.31}{#1}}
\newcommand{\PreprocessorTok}[1]{\textcolor[rgb]{0.68,0.00,0.00}{#1}}
\newcommand{\RegionMarkerTok}[1]{\textcolor[rgb]{0.00,0.23,0.31}{#1}}
\newcommand{\SpecialCharTok}[1]{\textcolor[rgb]{0.37,0.37,0.37}{#1}}
\newcommand{\SpecialStringTok}[1]{\textcolor[rgb]{0.13,0.47,0.30}{#1}}
\newcommand{\StringTok}[1]{\textcolor[rgb]{0.13,0.47,0.30}{#1}}
\newcommand{\VariableTok}[1]{\textcolor[rgb]{0.07,0.07,0.07}{#1}}
\newcommand{\VerbatimStringTok}[1]{\textcolor[rgb]{0.13,0.47,0.30}{#1}}
\newcommand{\WarningTok}[1]{\textcolor[rgb]{0.37,0.37,0.37}{\textit{#1}}}

\providecommand{\tightlist}{%
  \setlength{\itemsep}{0pt}\setlength{\parskip}{0pt}}\usepackage{longtable,booktabs,array}
\usepackage{calc} % for calculating minipage widths
% Correct order of tables after \paragraph or \subparagraph
\usepackage{etoolbox}
\makeatletter
\patchcmd\longtable{\par}{\if@noskipsec\mbox{}\fi\par}{}{}
\makeatother
% Allow footnotes in longtable head/foot
\IfFileExists{footnotehyper.sty}{\usepackage{footnotehyper}}{\usepackage{footnote}}
\makesavenoteenv{longtable}
\usepackage{graphicx}
\makeatletter
\def\maxwidth{\ifdim\Gin@nat@width>\linewidth\linewidth\else\Gin@nat@width\fi}
\def\maxheight{\ifdim\Gin@nat@height>\textheight\textheight\else\Gin@nat@height\fi}
\makeatother
% Scale images if necessary, so that they will not overflow the page
% margins by default, and it is still possible to overwrite the defaults
% using explicit options in \includegraphics[width, height, ...]{}
\setkeys{Gin}{width=\maxwidth,height=\maxheight,keepaspectratio}
% Set default figure placement to htbp
\makeatletter
\def\fps@figure{htbp}
\makeatother
% definitions for citeproc citations
\NewDocumentCommand\citeproctext{}{}
\NewDocumentCommand\citeproc{mm}{%
  \begingroup\def\citeproctext{#2}\cite{#1}\endgroup}
\makeatletter
 % allow citations to break across lines
 \let\@cite@ofmt\@firstofone
 % avoid brackets around text for \cite:
 \def\@biblabel#1{}
 \def\@cite#1#2{{#1\if@tempswa , #2\fi}}
\makeatother
\newlength{\cslhangindent}
\setlength{\cslhangindent}{1.5em}
\newlength{\csllabelwidth}
\setlength{\csllabelwidth}{3em}
\newenvironment{CSLReferences}[2] % #1 hanging-indent, #2 entry-spacing
 {\begin{list}{}{%
  \setlength{\itemindent}{0pt}
  \setlength{\leftmargin}{0pt}
  \setlength{\parsep}{0pt}
  % turn on hanging indent if param 1 is 1
  \ifodd #1
   \setlength{\leftmargin}{\cslhangindent}
   \setlength{\itemindent}{-1\cslhangindent}
  \fi
  % set entry spacing
  \setlength{\itemsep}{#2\baselineskip}}}
 {\end{list}}
\usepackage{calc}
\newcommand{\CSLBlock}[1]{\hfill\break\parbox[t]{\linewidth}{\strut\ignorespaces#1\strut}}
\newcommand{\CSLLeftMargin}[1]{\parbox[t]{\csllabelwidth}{\strut#1\strut}}
\newcommand{\CSLRightInline}[1]{\parbox[t]{\linewidth - \csllabelwidth}{\strut#1\strut}}
\newcommand{\CSLIndent}[1]{\hspace{\cslhangindent}#1}

\KOMAoption{captions}{tableheading}
\makeatletter
\@ifpackageloaded{caption}{}{\usepackage{caption}}
\AtBeginDocument{%
\ifdefined\contentsname
  \renewcommand*\contentsname{Table of contents}
\else
  \newcommand\contentsname{Table of contents}
\fi
\ifdefined\listfigurename
  \renewcommand*\listfigurename{List of Figures}
\else
  \newcommand\listfigurename{List of Figures}
\fi
\ifdefined\listtablename
  \renewcommand*\listtablename{List of Tables}
\else
  \newcommand\listtablename{List of Tables}
\fi
\ifdefined\figurename
  \renewcommand*\figurename{Figure}
\else
  \newcommand\figurename{Figure}
\fi
\ifdefined\tablename
  \renewcommand*\tablename{Table}
\else
  \newcommand\tablename{Table}
\fi
}
\@ifpackageloaded{float}{}{\usepackage{float}}
\floatstyle{ruled}
\@ifundefined{c@chapter}{\newfloat{codelisting}{h}{lop}}{\newfloat{codelisting}{h}{lop}[chapter]}
\floatname{codelisting}{Listing}
\newcommand*\listoflistings{\listof{codelisting}{List of Listings}}
\makeatother
\makeatletter
\makeatother
\makeatletter
\@ifpackageloaded{caption}{}{\usepackage{caption}}
\@ifpackageloaded{subcaption}{}{\usepackage{subcaption}}
\makeatother
\ifLuaTeX
  \usepackage{selnolig}  % disable illegal ligatures
\fi
\usepackage{bookmark}

\IfFileExists{xurl.sty}{\usepackage{xurl}}{} % add URL line breaks if available
\urlstyle{same} % disable monospaced font for URLs
\hypersetup{
  pdftitle={Final Project Report},
  pdfauthor={Wanqi Yuan (wy21) \& Jonathan Gan (wg18)},
  colorlinks=true,
  linkcolor={blue},
  filecolor={Maroon},
  citecolor={Blue},
  urlcolor={Blue},
  pdfcreator={LaTeX via pandoc}}

\title{Final Project Report}
\author{Wanqi Yuan (wy21) \& Jonathan Gan (wg18)}
\date{Tue., Apr.~30}

\begin{document}
\maketitle

\section{Introduction}\label{introduction}

\subsection{Problem Statement}\label{problem-statement}

The parking garage problem is currently structured to assume that demand
is deterministic and constant. This is not representative of real world
conditions as demand is highly dynamic and will change in response to
various city conditions. It does not consider that the price of a
parking spot very much affects the demand for parking. That is, the
higher the price of a parking spot, the lower the demand for the spots
in that garage. The parking garage problem also currently will
``decide'' to add another level if demand exceeds capacity at all, but
that could be at a margin of even one spot, which is not realistic nor
profitable.

For our final project, we want to better capture the variance in demand
for parking, particularly in San Francisco. We will explore through
research the relationship between price of parking and the demand of
parking to implement into a sequential decision problem of whether to
build additional levels of a parking garage at a yearly timestep.

\subsection{Selected Feature}\label{selected-feature}

The feature we will include is the relationship between price and demand
and how a difference in price will affect total revenue. At each yearly
timestep, we will find the optimal amount of levels for the parking
garage to exist at to maximize revenue, operating under the assumption
that we are always building a capacity to match the demand. This
resolves our problem statement by providing a representation that demand
is not deterministic and that price is constant regardless of demand.

Demand is a quantifiable variable for many cases outside of parking that
determine whether a given action will be taken. By more accurately
modeling this relationship between how the price of \emph{something}
might change the demand of it, and therefore the revenue and whether the
construction that might take place as a result, we can apply this to
other climate scenarios that would require sequential decision making.
For example, the decision to expand a solar farm on an annual basis
depends on the demand of the energy for the clients it serves and the
price of the energy that can be supplied from the solar farm.

\section{Literature Review}\label{literature-review}

Much of existing literature explores the demand side of the problem,
using pricing to control demand instead of increasing capacity since
varying the capacity-side is often seen as impractical and
cost-prohibitive. For instance, Simicevic et al.~(2012) utilized a
binary logit model to identify the main parameters that influenced user
behavior in choosing where to park in Belgrade. They found that in the
case of parking garages, as parking cost increases, demand for parking
for the average driver would correspondingly decrease. This study proved
sensitivity of pricing towards driver behavior, which directly
correlates with garage occupancy.

While we recognize that expanding the capacity of parking garages is not
always practical, we wanted to incorporate this as an alternative since
we wanted to account for all the decisions of the garage owner. For
many, the main objective is to maximize profits, and in dense urban
areas with high demand, the best way to maximize revenue may be to
increase capacity. By incorporating capacity expansion, we aim to create
a more comprehensive model for this sequential decision process,
evaluating the tradeoffs between initial capital investment and long
term revenue gain from increased parking capacity.

Nijsten (2017) explored the optimization of parking capacities in urban
areas with the following objectives - minimizing emissions, travel
times, and maximizing efficiency in land utilization. Various
evolutionary algorithms such as mutation and survivor selection were
utilized to solve the optimization problem, applying to small networks.
It was found that a rank-based selection method performed the best in
optimizing parking capacities for a case study conducted in Delft. In
the determination of the initial capacity for construction within our
model, we based it on a similar ranking selection method to better model
the optimization process.

\section{Methodology}\label{methodology}

\subsection{Implementation}\label{implementation}

The data used to create our demand curve is sourced from the San
Francisco Municipal Transit Authority. We utilized entry and exit data
at the Union Square parking garage on March 29 2013 to determine demand
for parking in relation to the price, as well as garage occupancy. We
are assuming that conditions on this day are representative of daily
conditions on any given day in the year.

In order to build our price vs.~demand curve, many assumptions were
made. We took the number of entries/exits to be the total daily demand.
We then aimed to calculate a weighted average price for this total
demand. We did this by categorizing every entry/exit data point into
hourly price buckets varying from \$1-\$7. For example, if one of our
entries' price was between \$3.5-\$4.49, it would be categorized as a
\$4 cost. We calculated the average price to be the sum of each bucket
price times the number of data points within that bucket, divided by the
sum of all the data points, 657. This resulting price was at an hourly
time step, so we multiplied this price by 12(assuming that very few
people would park for more than half a day), and 365, to get it to a
yearly time step price. We also assumed based on the data from San
Francisco Municipal Transit Authority that at \$30/hour, the demand for
parking would be zero. With these assumptions, we derived the following
linear relationship: demand = -0.046*price + 1500.

From there, we built a function to optimize the number of levels built
starting with demand vs.~price relationship at a given year. We start
with time year 1, with zero levels built. Within each time step, an
original price is taken and at that associated demand, the total revenue
is calculated. Included in this analysis is the construction costs per
space, which is taken into account when calculating revenue. Litman
(2023) found average construction costs for basic parking structures for
various US cities in 2022. In San Francisco, it would cost \$30,316 to
construct one additional space in an existing parking structure. You
then consider within this analysis whether adding one more level that
includes 150 more spots results in more revenue. If it does, you build
that level. You then repeat the consideration for another level until it
is no longer more profitable, having found the optimal number of levels.

At the next time step, you start at the resulting number of levels from
the previous time step, but the process of deciding whether to build
additional levels remains the same. Additionally, at each time step, the
y-intercept increases at a certain percentage to represent overall
demand increase for parking. We took this value to be randomly generated
from a normal distribution that represents the population increase of
San Francisco, with a mean of 0.45\% and a standard deviation of 0.1\%.
This represents the uncertainty of population growth. The slope of the
curve is also changing to represent inflation at a certain rate, say,
4\%. That is, the slope represents how many spots of demand you lose, at
each dollar increases in price. At each time step, we divided the slope
by 1.04, to show how with inflation, the number of spots of demand you
lose with a dollar increase is less, because the dollar is worth less.
We continue this analysis for some number of years, but this can be done
for a shorter or longer period with a simple change.

\subsection{Validation}\label{validation}

In order to validate our code, we ran the simulation function with
varying values for deterministic parameters including: timestep, growth
increment, inflation rate, and construction cost. We did so to
qualitatively see if the results made sense with the change in
parameters. For example, decreasing the increment (the number of
additional spots per level built), should result in more levels being
built initially as well as more frequently as the years progress. The
same is true of the opposite. An increased inflation rate would mean
more levels being built sooner (since there would be less demand
decrease for each dollar decrease), which was reflected in how the code
ran. An increase in construction cost per space would also result in
less levels being built over time.

\section{Results}\label{results}

With the simulation, we ran the simulation for a total of 9 times,
varying each of the 3 main inputs to determine the relationship between
each input and the optimal number of levels to build per year.

\begin{Shaded}
\begin{Highlighting}[]
\ImportTok{using} \BuiltInTok{Plots}
\ImportTok{using} \BuiltInTok{Distributions}
\NormalTok{Plots.}\FunctionTok{default}\NormalTok{(; margin}\OperatorTok{=}\FloatTok{5}\NormalTok{Plots.mm)}
\end{Highlighting}
\end{Shaded}

\begin{Shaded}
\begin{Highlighting}[]
\KeywordTok{function} \FunctionTok{get\_profit}\NormalTok{(capacity, m, b, cost)}

    \CommentTok{\# Cost is cost per spot }
\NormalTok{    price }\OperatorTok{=} \FunctionTok{div}\NormalTok{(capacity }\OperatorTok{{-}}\NormalTok{ b, m) }\CommentTok{\# Floor Div for dollars}
    \ControlFlowTok{return}\NormalTok{ (price }\OperatorTok{*}\NormalTok{ capacity)}\FunctionTok{{-}}\NormalTok{(cost }\OperatorTok{*}\NormalTok{ capacity)}
\KeywordTok{end}
\end{Highlighting}
\end{Shaded}

\begin{verbatim}
get_profit (generic function with 1 method)
\end{verbatim}

\begin{Shaded}
\begin{Highlighting}[]
\KeywordTok{function} \FunctionTok{optimize}\NormalTok{(maxSpots, increment, start\_spots, m, cost\_per\_spot)}
\NormalTok{    b }\OperatorTok{=}\NormalTok{ maxSpots}
\NormalTok{    levels\_to\_add }\OperatorTok{=} \FunctionTok{div}\NormalTok{(maxSpots }\OperatorTok{{-}}\NormalTok{ start\_spots, increment)}
\NormalTok{    optLevels }\OperatorTok{=} \FloatTok{0} 
\NormalTok{    optProfit }\OperatorTok{=} \FunctionTok{get\_profit}\NormalTok{(start\_spots, m, b, cost\_per\_spot)}
\NormalTok{    curr\_levels }\OperatorTok{=} \FloatTok{0}
    \ControlFlowTok{while}\NormalTok{ curr\_levels }\OperatorTok{\textless{}}\NormalTok{ levels\_to\_add}
\NormalTok{        curr\_capacity }\OperatorTok{=}\NormalTok{ start\_spots }\OperatorTok{+}\NormalTok{ curr\_levels }\OperatorTok{*}\NormalTok{ increment}
\NormalTok{        profit }\OperatorTok{=} \FunctionTok{get\_profit}\NormalTok{(curr\_capacity, m, b, cost\_per\_spot)}
        \ControlFlowTok{if}\NormalTok{ profit }\OperatorTok{\textgreater{}}\NormalTok{ optProfit}
\NormalTok{            optProfit }\OperatorTok{=}\NormalTok{ profit}
\NormalTok{            optLevels }\OperatorTok{=}\NormalTok{ curr\_levels}
        \ControlFlowTok{end}
\NormalTok{        curr\_levels }\OperatorTok{+=} \FloatTok{1}
    \ControlFlowTok{end}
    \ControlFlowTok{return}\NormalTok{ optLevels}
\KeywordTok{end}
\end{Highlighting}
\end{Shaded}

\begin{verbatim}
optimize (generic function with 1 method)
\end{verbatim}

\begin{Shaded}
\begin{Highlighting}[]
\KeywordTok{function} \FunctionTok{simulation}\NormalTok{(timesteps, start\_spots, increment, demand\_growth\_rate, init\_b, m, inflation\_rate, cost\_per\_spot)}
\NormalTok{    y\_levels }\OperatorTok{=}\NormalTok{ [] }\CommentTok{\# LEVELS AT TIMETSTEP T}
\NormalTok{    y\_curr\_profit }\OperatorTok{=}\NormalTok{ [] }\CommentTok{\# PROFIT AT TIMESTEP T}
\NormalTok{    y\_net\_profit }\OperatorTok{=}\NormalTok{ [] }\CommentTok{\# NET PROFITS}
\NormalTok{    net\_profit }\OperatorTok{=} \FloatTok{0}
\NormalTok{    curr\_levels }\OperatorTok{=} \FloatTok{0}
\NormalTok{    timestep }\OperatorTok{=} \FloatTok{0}
\NormalTok{    curr\_spots }\OperatorTok{=}\NormalTok{ start\_spots}
\NormalTok{    b }\OperatorTok{=}\NormalTok{ init\_b}

    \ControlFlowTok{while}\NormalTok{ timestep }\OperatorTok{\textless{}}\NormalTok{ timesteps}
\NormalTok{        levels\_to\_add }\OperatorTok{=} \FunctionTok{optimize}\NormalTok{(b, increment, curr\_spots, m, cost\_per\_spot)}
        \ControlFlowTok{if}\NormalTok{ levels\_to\_add }\OperatorTok{==} \FloatTok{0}
            \CommentTok{\# println("Whoops")}
        \ControlFlowTok{end}
\NormalTok{        curr\_spots }\OperatorTok{+=}\NormalTok{ increment }\OperatorTok{*}\NormalTok{ levels\_to\_add}
\NormalTok{        curr\_levels }\OperatorTok{+=}\NormalTok{ levels\_to\_add}

\NormalTok{        curr\_profit }\OperatorTok{=} \FunctionTok{get\_profit}\NormalTok{(curr\_spots, m, b, cost\_per\_spot) }\OperatorTok{/} \FloatTok{1000}
\NormalTok{        net\_profit }\OperatorTok{+=}\NormalTok{ curr\_profit}
        \FunctionTok{push!}\NormalTok{(y\_curr\_profit, curr\_profit)}
        \FunctionTok{push!}\NormalTok{(y\_net\_profit, net\_profit)}
        \FunctionTok{push!}\NormalTok{(y\_levels, curr\_levels)}

\NormalTok{        b }\OperatorTok{*=}\NormalTok{ (}\FloatTok{1} \OperatorTok{+}\NormalTok{ demand\_growth\_rate)}
\NormalTok{        m }\OperatorTok{/=}\NormalTok{ (}\FloatTok{1} \OperatorTok{+}\NormalTok{ inflation\_rate)}

\NormalTok{        timestep }\OperatorTok{+=} \FloatTok{1} 
    \ControlFlowTok{end}

    \ControlFlowTok{return}\NormalTok{ y\_levels}
\KeywordTok{end}
\end{Highlighting}
\end{Shaded}

\begin{verbatim}
simulation (generic function with 1 method)
\end{verbatim}

INCREMENT = 100 Inflation Rate = 0.04 Construction Cost = 30000

\begin{Shaded}
\begin{Highlighting}[]
\NormalTok{timesteps }\OperatorTok{=} \FloatTok{20} 
\NormalTok{start\_spots }\OperatorTok{=} \FloatTok{0}
\NormalTok{increment }\OperatorTok{=} \FloatTok{100}
\NormalTok{demand\_growth\_rate }\OperatorTok{=} \FunctionTok{rand}\NormalTok{(}\FunctionTok{Normal}\NormalTok{(}\FloatTok{0.0045}\NormalTok{, }\FloatTok{0.001}\NormalTok{))}
\NormalTok{init\_b }\OperatorTok{=} \FloatTok{1500}
\NormalTok{m }\OperatorTok{=} \OperatorTok{{-}}\FloatTok{0.046}
\NormalTok{inflation\_rate }\OperatorTok{=} \FloatTok{0.04}
\NormalTok{cost\_per\_spot }\OperatorTok{=} \FloatTok{30000}
\NormalTok{y\_levels }\OperatorTok{=} \FunctionTok{simulation}\NormalTok{(timesteps, start\_spots, increment, demand\_growth\_rate, init\_b, m, inflation\_rate, cost\_per\_spot)}
\NormalTok{display\_growth\_rate }\OperatorTok{=} \FunctionTok{round}\NormalTok{(demand\_growth\_rate, sigdigits}\OperatorTok{=}\FloatTok{3}\NormalTok{)}
\NormalTok{plot\_title }\OperatorTok{=} \StringTok{"Parking Garage Simulation"}
\NormalTok{font\_size }\OperatorTok{=} \FloatTok{16}
\FunctionTok{println}\NormalTok{(y\_levels)}
\FunctionTok{plot}\NormalTok{(y\_levels, label}\OperatorTok{=}\StringTok{"Optimal Levels at Time t"}\NormalTok{, xlabel}\OperatorTok{=}\StringTok{"Year"}\NormalTok{, ylabel}\OperatorTok{=}\StringTok{"Number of Levels"}\NormalTok{,}
\NormalTok{title}\OperatorTok{=}\NormalTok{plot\_title, titlefontsize}\OperatorTok{=}\NormalTok{font\_size, lw}\OperatorTok{=}\FloatTok{1}\NormalTok{, gridlinewidth}\OperatorTok{=}\FloatTok{1}\NormalTok{, xticks}\OperatorTok{=}\FloatTok{0}\OperatorTok{:}\FloatTok{1}\OperatorTok{:}\FloatTok{25}\NormalTok{, yticks}\OperatorTok{=}\FloatTok{0}\OperatorTok{:}\FloatTok{1}\OperatorTok{:}\FloatTok{100}\NormalTok{)}
\CommentTok{\# plot(y\_curr\_profit, label="Current Annual Profit at time t")}
\CommentTok{\# plot(y\_net\_profit, label="Net Profit at time t")}
\CommentTok{\# legend()}
\CommentTok{\#display()}
\end{Highlighting}
\end{Shaded}

\begin{verbatim}
Any[1, 1, 1, 1, 2, 2, 2, 3, 3, 3, 3, 3, 4, 4, 4, 4, 5, 5, 5, 5]
\end{verbatim}

\includegraphics{template_files/mediabag/template_files/figure-pdf/cell-6-output-2.pdf}

INCREMENT = 150 Inflation Rate = 0.04 Construction Cost = 30000

\begin{verbatim}
Any[0, 1, 1, 1, 1, 1, 2, 2, 2, 2, 2, 2, 3, 3, 3, 3, 3, 3, 3, 3]
\end{verbatim}

\includegraphics{template_files/mediabag/template_files/figure-pdf/cell-7-output-2.pdf}

INCREMENT = 200 Inflation Rate = 0.04 Construction Cost = 30000

\begin{verbatim}
Any[0, 0, 1, 1, 1, 1, 1, 1, 1, 2, 2, 2, 2, 2, 2, 2, 2, 2, 2, 3]
\end{verbatim}

\includegraphics{template_files/mediabag/template_files/figure-pdf/cell-8-output-2.pdf}

Increment = 150 INFLATION RATE = 0.03 Construction Cost = 30000

\begin{Shaded}
\begin{Highlighting}[]
\NormalTok{timesteps }\OperatorTok{=} \FloatTok{20} 
\NormalTok{start\_spots }\OperatorTok{=} \FloatTok{0}
\NormalTok{increment }\OperatorTok{=} \FloatTok{150}
\NormalTok{demand\_growth\_rate }\OperatorTok{=} \FunctionTok{rand}\NormalTok{(}\FunctionTok{Normal}\NormalTok{(}\FloatTok{0.0045}\NormalTok{, }\FloatTok{0.001}\NormalTok{))}
\NormalTok{init\_b }\OperatorTok{=} \FloatTok{1500}
\NormalTok{m }\OperatorTok{=} \OperatorTok{{-}}\FloatTok{0.046}
\NormalTok{inflation\_rate }\OperatorTok{=} \FloatTok{0.03}
\NormalTok{cost\_per\_spot }\OperatorTok{=} \FloatTok{30000}
\NormalTok{y\_levels }\OperatorTok{=} \FunctionTok{simulation}\NormalTok{(timesteps, start\_spots, increment, demand\_growth\_rate, init\_b, m, inflation\_rate, cost\_per\_spot)}
\NormalTok{display\_growth\_rate }\OperatorTok{=} \FunctionTok{round}\NormalTok{(demand\_growth\_rate, sigdigits}\OperatorTok{=}\FloatTok{3}\NormalTok{)}
\NormalTok{plot\_title }\OperatorTok{=} \StringTok{"Parking Garage Simulation"}
\NormalTok{font\_size }\OperatorTok{=} \FloatTok{16}
\FunctionTok{println}\NormalTok{(y\_levels)}
\FunctionTok{plot}\NormalTok{(y\_levels, label}\OperatorTok{=}\StringTok{"Optimal Levels at Time t"}\NormalTok{, xlabel}\OperatorTok{=}\StringTok{"Year"}\NormalTok{, ylabel}\OperatorTok{=}\StringTok{"Number of Levels"}\NormalTok{,}
\NormalTok{title}\OperatorTok{=}\NormalTok{plot\_title, titlefontsize}\OperatorTok{=}\NormalTok{font\_size, lw}\OperatorTok{=}\FloatTok{1}\NormalTok{, gridlinewidth}\OperatorTok{=}\FloatTok{1}\NormalTok{, xticks}\OperatorTok{=}\FloatTok{0}\OperatorTok{:}\FloatTok{1}\OperatorTok{:}\FloatTok{25}\NormalTok{, yticks}\OperatorTok{=}\FloatTok{0}\OperatorTok{:}\FloatTok{1}\OperatorTok{:}\FloatTok{100}\NormalTok{)}
\CommentTok{\# plot(y\_curr\_profit, label="Current Annual Profit at time t")}
\CommentTok{\# plot(y\_net\_profit, label="Net Profit at time t")}
\CommentTok{\# legend()}
\CommentTok{\#display()}
\end{Highlighting}
\end{Shaded}

\begin{verbatim}
Any[0, 1, 1, 1, 1, 1, 1, 1, 2, 2, 2, 2, 2, 2, 2, 2, 3, 3, 3, 3]
\end{verbatim}

\includegraphics{template_files/mediabag/template_files/figure-pdf/cell-9-output-2.pdf}

Increment = 150 INFLATION RATE = 0.04 Construction Cost = 30000

\begin{verbatim}
Any[0, 1, 1, 1, 1, 1, 1, 2, 2, 2, 2, 2, 2, 2, 3, 3, 3, 3, 3, 3]
\end{verbatim}

\includegraphics{template_files/mediabag/template_files/figure-pdf/cell-10-output-2.pdf}

Increment = 150 INFLATION RATE = 0.05 Construction Cost = 30000

\begin{verbatim}
Any[0, 1, 1, 1, 1, 1, 2, 2, 2, 2, 2, 3, 3, 3, 3, 3, 3, 3, 3, 4]
\end{verbatim}

\includegraphics{template_files/mediabag/template_files/figure-pdf/cell-11-output-2.pdf}

Increment = 150 Inflation Rate = 0.04 CONSTRUCTION COST = 20000

\begin{Shaded}
\begin{Highlighting}[]
\NormalTok{timesteps }\OperatorTok{=} \FloatTok{20} 
\NormalTok{start\_spots }\OperatorTok{=} \FloatTok{0}
\NormalTok{increment }\OperatorTok{=} \FloatTok{150}
\NormalTok{demand\_growth\_rate }\OperatorTok{=} \FunctionTok{rand}\NormalTok{(}\FunctionTok{Normal}\NormalTok{(}\FloatTok{0.0045}\NormalTok{, }\FloatTok{0.001}\NormalTok{))}
\NormalTok{init\_b }\OperatorTok{=} \FloatTok{1500}
\NormalTok{m }\OperatorTok{=} \OperatorTok{{-}}\FloatTok{0.046}
\NormalTok{inflation\_rate }\OperatorTok{=} \FloatTok{0.04}
\NormalTok{cost\_per\_spot }\OperatorTok{=} \FloatTok{20000}
\NormalTok{y\_levels }\OperatorTok{=} \FunctionTok{simulation}\NormalTok{(timesteps, start\_spots, increment, demand\_growth\_rate, init\_b, m, inflation\_rate, cost\_per\_spot)}
\NormalTok{display\_growth\_rate }\OperatorTok{=} \FunctionTok{round}\NormalTok{(demand\_growth\_rate, sigdigits}\OperatorTok{=}\FloatTok{3}\NormalTok{)}
\NormalTok{plot\_title }\OperatorTok{=} \StringTok{"Parking Garage Simulation"}
\NormalTok{font\_size }\OperatorTok{=} \FloatTok{16}
\FunctionTok{println}\NormalTok{(y\_levels)}
\FunctionTok{plot}\NormalTok{(y\_levels, label}\OperatorTok{=}\StringTok{"Optimal Levels at Time t"}\NormalTok{, xlabel}\OperatorTok{=}\StringTok{"Year"}\NormalTok{, ylabel}\OperatorTok{=}\StringTok{"Number of Levels"}\NormalTok{,}
\NormalTok{title}\OperatorTok{=}\NormalTok{plot\_title, titlefontsize}\OperatorTok{=}\NormalTok{font\_size, lw}\OperatorTok{=}\FloatTok{1}\NormalTok{, gridlinewidth}\OperatorTok{=}\FloatTok{1}\NormalTok{, xticks}\OperatorTok{=}\FloatTok{0}\OperatorTok{:}\FloatTok{1}\OperatorTok{:}\FloatTok{25}\NormalTok{, yticks}\OperatorTok{=}\FloatTok{0}\OperatorTok{:}\FloatTok{1}\OperatorTok{:}\FloatTok{100}\NormalTok{)}
\CommentTok{\# plot(y\_curr\_profit, label="Current Annual Profit at time t")}
\CommentTok{\# plot(y\_net\_profit, label="Net Profit at time t")}
\CommentTok{\# legend()}
\CommentTok{\#display()}
\end{Highlighting}
\end{Shaded}

\begin{verbatim}
Any[2, 2, 2, 2, 2, 3, 3, 3, 3, 3, 3, 3, 3, 3, 4, 4, 4, 4, 4, 4]
\end{verbatim}

\includegraphics{template_files/mediabag/template_files/figure-pdf/cell-12-output-2.pdf}

Increment = 150 Inflation Rate = 0.04 CONSTRUCTION COST = 30000

\begin{verbatim}
Any[0, 1, 1, 1, 1, 1, 1, 2, 2, 2, 2, 2, 2, 2, 3, 3, 3, 3, 3, 3]
\end{verbatim}

\includegraphics{template_files/mediabag/template_files/figure-pdf/cell-13-output-2.pdf}

Increment = 150 Inflation Rate = 0.04 CONSTRUCTION COST = 40000

\begin{verbatim}
Any[0, 0, 0, 0, 0, 0, 0, 1, 1, 1, 1, 1, 2, 2, 2, 2, 2, 2, 3, 3]
\end{verbatim}

\includegraphics{template_files/mediabag/template_files/figure-pdf/cell-14-output-2.pdf}

\section{Conclusions}\label{conclusions}

\subsection{Discussion}\label{discussion}

Our results demonstrate the modeling of the relationship between price
and demand for a sequential decision problem where demand is a large
factor on a given decision. As mentioned briefly in the problem
statement, this model structure can be applied to many other cases
beyond the parking garage problem. It can be applied to analyses for
infrastructure development that supports renewable energy with the goal
of building the appropriate size for that infrastructure to support the
demands, remaining profitable, but also be resilient against natural
disasters. For example, the model could be developed to find the ``sweet
spot'' size for a solar farm where it is large enough to support the
demands of the area it is serving, still being profitable, while also
being large enough to remain partially operational should a natural
disaster result in part of the farming being temporarily under
maintenance. It can also provide more guidance on the front end of how
changes in population or economical changes can impact how they can
anticipate expansion of energy infrastructure. A large limitation to
this model is that the way we determine demand is not the most accurate.
We assumed that every entry/exit data point was equal to a full 12 hour
day of parking in order to build the curve, which is not true as the raw
data shows. We also aggregated all data to assume yearly demand and
price from one day, which is not accurate or representative. We would
suggest a better method to develop a price vs.~demand curve.

\subsection{Conclusions}\label{conclusions-1}

While the model structure allows for varying parameter changes, if we're
assuming that each level of the parking garage includes 150 spots, a
randomly sampled demand increase with a mean of 0.45\% and a standard
deviation of 0.1\%, and an interest rate of 4\%, we observed that on
average, a new level was built every 4-6 years. It was interesting to
see that at the starting demand, it was not profitable to build even the
first level until about 4-5 years in. This feature we've implemented now
demonstrates how the relationship between price and demand will affect
the decision at each yearly time step on whether to build an additional
level. It also shows how it might take years before it is profitable to
build a garage at all with a starting demand and price relationship
compared to the costs to build.

In the case of varying parameters, the lower the number of spots per
level, the more levels were built sooner since there was a lower
threshold of demand needed to justify the additional costs of building a
new level of spots. At a higher interest rate, it would also result in
building more levels sooner, because the demand decrease per dollar
increase, was less and less each year. And with higher construction
costs per space, the less levels were built.

Our findings illustrate the correlation between price and demand within
a sequential decision-making scenario, where demand significantly
influences the decision-making process. This modeling framework is
applicable to various scenarios beyond the context of parking garages.
It can be utilized in analyzing infrastructure development aimed at
supporting renewable energy initiatives, ensuring that infrastructure
sizes align with demands to maintain profitability while also enhancing
resilience against natural disasters.

\section{References}\label{references}

Litman, T. (2023). (rep.). Comprehensive Parking Supply, Cost and
Pricing Analysis. Nijsten, T. (2023, August 25). Optimizing Parking
Capacities in Urban Areas (thesis). XCarCity NL. Retrieved April 26,
2024, from https://xcarcity.nl/downloads/MSc\_thesis\_Tygo\_Nijsten.pdf.
Simićević, J., Milosavljević, N., \& Maletić, G. (2012). Influence of
parking price on parking garage users' behaviour. PROMET - Traffic \&
Transportation, 24(5), 413--423. https://doi.org/10.7307/ptt.v24i5.1177

\phantomsection\label{refs}
\begin{CSLReferences}{0}{1}
\end{CSLReferences}



\end{document}
